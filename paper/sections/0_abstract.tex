Street network analysis is widely used to relate urban form to land-use intensity and travel behaviour, yet ``closeness'' centrality is inconsistently described, with cited formulas diverging from those implemented in software packages. This matters for contemporary analysis, where centralities are increasingly computed in localised (distance-threshold) forms to reduce edge effects and enable multi-scalar comparisons. In localised analysis, reachable node counts vary by origin. While \emph{Closeness} and \emph{Normalised Closeness} are effective for comparing nodes within a fixed graph, they behave counter-intuitively across differently sized sub-graphs due to scaling effects associated with varying numbers of nodes. We clarify this mechanism, demonstrate reliable alternatives (\emph{Improved} and \emph{Harmonic} Closeness), and provide an openly reproducible workflow.

We test implications in Madrid, Spain, using street-network, land-use, and travel-survey data. We compute multiple formulations, including \emph{Closeness}, \emph{Normalised Closeness}, \emph{Improved}, \emph{Harmonic}, and \emph{Gravity}, for metric and angular distances across multiple distance thresholds, and relate them to land-use accessibility and trip volumes using Spearman correlations with spatially robust (block-bootstrap) uncertainty estimates. \emph{Closeness} shows negative associations with land-use intensity, while \emph{Normalised Closeness} shows weak associations. In contrast, \emph{Improved}, \emph{Harmonic}, and \emph{Gravity} remain stable and consistently positively associated across scales, consistent with their adoption in commonly used software packages. We conclude with practical guidance: prioritise formulations that scale appropriately with varying sub-graph sizes, and clearly document the precise formulations and workflows to support reproducibility and comparability of methods.
