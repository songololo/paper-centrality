Street network analysis is widely used to relate urban form to land-use intensity and travel behaviour, yet ``closeness'' centrality is described and implemented inconsistently across the literature, with formulas cited in papers often differing from those computed by widely used packages. This discrepancy matters for contemporary practice, where centralities are increasingly computed in local (distance-threshold) analyses to reduce edge effects and enable multi-scale comparisons, in which case the number of reachable nodes varies from location to location. This paper explains why standard mathematical closeness, and its common normalisation, can behave counter-intuitively under localised analysis, and provides an openly reproducible workflow to support comparability across studies.

The implications are tested empirically using open street-network, land-use premises, and origin--destination travel-survey data for Madrid, Spain. Multiple closeness formulations---including classic, normalised, and \emph{Improved} variants, alongside \emph{Harmonic Closeness} and \emph{Gravity} alternatives, are computed for metric and angular distances across radii from 500m to 10km and correlated with land-use accessibility and trip volumes using Spearman rank coefficients with spatially robust (block-bootstrap) uncertainty estimates. Results show that classic closeness can yield sign reversals (strong negative associations with land-use intensity) and normalised closeness is often weakly associated, whereas \emph{Improved}, \emph{Harmonic}, and \emph{Gravity} formulations are stable and consistently positively associated across scales. The paper concludes with practical guidance: for localised street network analysis, prefer formulations that scale appropriately with varying sub-graph sizes, and clearly document the precise formulations and workflows used to support reproducibility and comparability of methods.
