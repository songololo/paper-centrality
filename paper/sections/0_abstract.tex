Closeness centrality formulas cited in the street network analysis literature diverge from those implemented in computational packages, creating a barrier to reproducibility. A researcher implementing the commonly cited \emph{Normalised Closeness} formula will obtain different results from studies using packages such as \texttt{Depthmap}, \texttt{Place Syntax Tool}, or \texttt{cityseer}, which implement variants such as \emph{Improved Closeness}. This distinction matters for localised (distance-threshold) analysis, now standard practice, because subgraph sizes vary by origin. Under these conditions, mathematical \emph{Closeness} exhibits sign reversal, while \emph{Normalised Closeness} yields negligible associations. \emph{Improved} and \emph{Harmonic Closeness} scale correctly: the former divides by \textbf{average} rather than \textbf{total} distance; the latter sums distance reciprocals, achieving similar scaling through a different mechanism.

We demonstrate these differences empirically in Madrid, Spain, using correlations with land-use intensity and trip counts as a diagnostic tool. Across distance thresholds from 500m to 10km, the formulations diverge substantially: \emph{Closeness} shows negative associations ($\rho \approx -0.70$); \emph{Normalised Closeness} shows negligible associations ($\rho \approx -0.11$); \emph{Improved}, \emph{Harmonic}, and \emph{Gravity} formulations show positive associations ($\rho \approx +0.67$). This divergence pattern persists across metric and angular distances, length-weighted variants, and origin-destination trip data. The pattern of divergence---not the correlation magnitudes---is the finding; the mathematical argument is general, while specific correlations are context-dependent.

We recommend that practitioners use closeness formulations appropriate for localised analysis and document computational methods precisely. Code and data are openly provided.
