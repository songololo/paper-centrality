\section{Discussion and Conclusions}\label{discussion}

The empirical results for Madrid are consistent with the theoretical hypothesis. Regarding \textbf{RQ1}, we find a meaningful distinction: mathematical \emph{Closeness} exhibits sign reversal when applied to localised analysis, correlating negatively with urban intensity where positive associations are expected. \emph{Normalised Closeness} does not correct this---it yields negligible associations. By contrast, \emph{Improved Closeness} and \emph{Harmonic Closeness} scale as expected with varying subgraph sizes. Regarding \textbf{RQ2}, these behavioural differences manifest empirically: results hold for land-use accessibility and trip counts, for length-weighted and unweighted variants, and for both metric and geometric distances.

\subsection{Who Is Affected}

Software developers have addressed this issue in practice. \texttt{DepthmapX} \cite{turner_depthmapx_2020}, \texttt{Place Syntax Tool} \cite{stahle_place_2023}, and \texttt{cityseer} \cite{simons_cityseer_2023} implement \emph{Improved Closeness} or similar variants, and some offer \emph{Harmonic Closeness} and \emph{Gravity} measures.\footnote{Some packages also provide \emph{Normalised Closeness} as an option.} Researchers using these tools obtain correct results even when citing different formulations.

The reproducibility problem affects researchers who:
\begin{itemize}
  \item Implement closeness centrality from cited formulas using generic network packages;\footnote{NetworkX provides \texttt{harmonic\_centrality} and \texttt{closeness\_centrality} with a \texttt{wf\_improved} parameter (enabled by default) implementing the Wasserman-Faust formulation, which behaves appropriately for disconnected subgraphs.}
  \item Attempt to replicate published studies without access to original code;
  \item Review or teach network centrality methods based on commonly cited formulations.
\end{itemize}

In these cases, implementing the cited \emph{Normalised Closeness} formula for localised analysis will yield different results from the original study---potentially with reversed sign or negligible magnitude.

\subsection{Implications for the Literature}

A documentation gap exists between what papers cite and what software computes. This gap likely persists because studies using appropriate software implementations produce expected results, even when citing different formulations. The mismatch becomes consequential only when attempting replication or when using generic tools.

This has implications for interpreting existing literature:
\begin{itemize}
  \item Studies using \texttt{Depthmap} or similar tools likely produced valid results regardless of cited formulations;
  \item Studies using generic network packages or custom implementations warrant scrutiny regarding which formula was actually computed;
  \item Comparative evaluations of closeness formulations should verify that implementations match cited formulas.
\end{itemize}

Earlier studies that cite \emph{Normalised Closeness} without providing open code, that select small geographic areas, or that do not buffer boundaries to control edge effects \cite{Turner2007, Porta2006} may warrant re-examination---not because their results are necessarily wrong, but because the relationship between cited methods and computed results is unclear.

\subsection{Practical Guidance}

For localised street network analysis, we recommend:
\begin{enumerate}
  \item Use closeness formulations that scale with varying subgraph sizes: \emph{Improved Closeness}, \emph{Harmonic Closeness}, or \emph{Gravity}-based measures;
  \item Document the precise formulation used, not just the generic term `closeness';
  \item Provide code or specify the software package and version to enable replication;
  \item When using generic network packages, verify that the closeness implementation is appropriate for localised analysis.
\end{enumerate}

\subsection{Limitations}

This study uses a single city (Madrid), which is appropriate because our argument is mathematical: the theoretical reasons why formulations diverge under localised analysis apply to any network where subgraph sizes vary by origin. The Madrid case demonstrates that this divergence has empirically detectable consequences. Correlation magnitudes will differ across urban contexts (e.g., cities with linear high streets, polycentric structures, or car-oriented development), but our contribution concerns the existence and direction of formulation differences, not their precise magnitude. The empirical analysis is diagnostic and cross-sectional: correlations reveal that formulations behave differently, not that any particular formulation correctly captures urban accessibility. We do not model causal relationships between centrality and land-use.

\subsection{Conclusion}

The divergence between cited closeness formulations and software implementations creates a reproducibility barrier in street network analysis. Mathematical \emph{Closeness} exhibits sign reversal under localised analysis; \emph{Normalised Closeness} yields negligible associations; \emph{Improved} and \emph{Harmonic Closeness} scale as expected under varying subgraph sizes. Researchers should use appropriate formulations for localised analysis and document computational methods precisely. We provide open code and data to support reproducibility and to serve as a reference for future methodological comparisons.
