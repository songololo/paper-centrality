\section{Discussion and Conclusions}\label{discussion}

The results of the empirical observations for the Madrid case confirm the behaviour anticipated by the theoretical hypothesis. Mathematically defined \emph{Closeness} behaves opposite to its intended usage when applied to localised street network analysis. When normalised in the form of \emph{Normalised Closeness}, it performs poorly, especially when compared to localised variants of closeness centrality such as \emph{Improved Closeness} and \emph{Harmonic Closeness}. The results are reflected for access to landuses as well as for origin-destination trip counts, regardless of whether closeness centralities are length weighted or not, and irrespective of whether \emph{metric} or \emph{geometric} distances are used.

Developers of domain-specific urban network analysis packages appear to be aware of this issue even if not reflected in wider discussion; the \texttt{DepthmapX} \cite{turner_depthmapx_2020}, \texttt{Place Syntax Toolkit} \cite{stahle_place_2023}, and \texttt{cityseer} \cite{simons_cityseer_2023} packages all use or offer a form of \emph{Improved Closeness} \cite{Wasserman1994}, and in some cases offer other suitable variants such as \emph{Harmonic Closeness} and the \emph{Gravity Index}.\footnote{Some packages also provide \emph{Normalised Closeness} as an option.}

We suggest the following implications:
\begin{itemize}
  \item There appears to be a misunderstanding in the general street network analysis literature that \emph{Normalised Closeness} is being used for localised forms of street network analysis. Localised forms of analysis are used by convention and are considered best practice for reasons explained in Section~\ref{network-analysis}.
  \item We suspect that one of the reasons this misattribution persists is that even though \emph{Normalised Closeness} is typically cited by the authors of papers, these studies are often relying on computational packages which use forms of closeness that work for localised analysis. Consequently, outputs would typically perform as expected even if users were under mistaken impressions regarding the form of closeness being used.
  \item In cases where computational workflows relying on generic network analysis packages or non-open-source packages are used, this raises wider issues of reproducibility. On one hand, researchers trying to replicate the behaviour of existing studies are faced with a conundrum because they may be trying to replicate work using an ineffective form of closeness centrality. Secondly, where the code is not openly available, there may be issues regarding the reliability or interpretability of findings because it may not be clear whether the cited formulation matches that used for the calculation process.
\end{itemize}

These ambiguities warrant further investigation into the robustness of currently held assumptions regarding the relative performance of different centralities and forms of distance heuristic, which have been influenced by widely cited earlier studies. Some such studies cite \emph{Normalised Closeness} but do not provide open code to confirm the computational methods used, select small geographic areas for analysis, and do not buffer boundaries to control for edge effects (which would skew closeness centralities). The conclusions drawn in such cases regarding the relative performance of closeness measures may warrant re-examination \cite{Turner2007, Porta2006}.

We consequently encourage the use of openly available reference datasets and reproducible workflows, as we have made available in this study, so that different measures can be benchmarked openly and collaboratively against common points of reference. This would facilitate the evaluation of newly available methods against established variants while accounting for differences which might otherwise be attributable to variations in urban context or data representation. It would also help uncover situations such as the widespread misconceptions on closeness centralities discussed in this study.

In conclusion, we recommend that researchers working with localised street network analysis use \emph{Improved Closeness}, \emph{Harmonic Closeness}, or similar variants rather than \emph{Normalised Closeness}, and that they clearly document and share the methods and precise computational formulations used in their analyses.
