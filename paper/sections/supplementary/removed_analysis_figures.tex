REMOVED

\subsection{Analysis Figures}\label{analysis-figures}

\begin{figure*}[htb]
 \centering
 \includegraphics[width=\linewidth, keepaspectratio]{plots/cent_corr_matrix_betw.pdf}
 \caption{Correlation grid comparing correlations for betweenness centralities. Notice the diminished contrast for NACH measures across smaller and larger distance radii.}\label{fig:cent_corr_matrix_betw}
\end{figure*}

The following Figures represent correlation grids comparing network centralities to landuse accessibilities. They can be interpreted as follows:
\begin{itemize}
    \item The $y$ axis labels correspond to a network centrality measure.
    \item The $x$ axis labels correspond to the distance used for the localised catchment (so-called radii or moving windows) at which the network centrality measures shown on the $y$ axis have been computed, ranging from $500m$ to $10km$.
    \item The correlations shown in the individual cells correspond to the Spearman Rank correlation for a given network centrality measure ($y$ axis) at a given distance ($x$ axis) correlated to the landuse theme for a given correlation matrix.
    \item The landuse theme is identified by the title given to the respective correlation matrices, for example, ``PCA" for Principle Component Analysis or ``Retail" for accessibility to retail establishments.
    \item The overall figure titles provide additional context specific to a correlation matrix, such as whether landuse accessibilities are computed using a spatial impedance weighted form (where landuses farther from the origin contribute less to the total), or whether the centralities are computed in a street-length weighted form, or whether the cost parameter for measuring distances to landuses is metric distance or geometric (angular) distance.
    \item Where using weighted landuse accessibility counts corresponding to a maximum distance of $d_{max}=200m$, this represents a decay parameter of $\beta=0.02$ following the orange line shown in Figure~\ref{fig:beta_decays_3}. A landuse at $0m$ distance would therefore count 1 towards the summation whereas a landuse at $200m$ would count near zero. The average walking distance represented by the decaying spatial impedance curve is approximately $\mu=70m$. Framed differently, a landuse at $0m$ distance counting ``1" would contribute the same as approximately five landuses located at $80m$ distance, contributing roughly ``0.2" each.
    \item An unweighted summation for $100m$ distance (where all landuses within $100m$ are counted as 1 towards the total) is included for comparison.
\end{itemize}

