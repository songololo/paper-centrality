\section{Results}\label{data-analysis}

A visual survey of the 1000m closeness centralities is plotted for \emph{metric} distance measures in Figure~\ref{fig:closeness_compare}, with \emph{geometric} distance (angular) measures shown in Figure~\ref{fig:closeness_compare_ang}. The correlation matrix for the discussed closeness centralities is shown in Figure~\ref{fig:cent_corr_matrix_close}.

\subsection{Visual Comparisons}

As illustrated in Figure~\ref{fig:closeness_compare}, \emph{Closeness} demonstrates notable challenges due to its mathematical behaviour, which scales differently depending on the number of nodes under consideration. The iterative process of isolating catchments for localised calculations therefore causes \emph{Closeness} to perform opposite to its intended usage. \emph{Normalised Closeness} and \emph{NAIN} reduce the variance, but do not recover meaningful contrast across different street network intensities. The \emph{Improved}, \emph{Gravity}, and \emph{Harmonic} formulations behave as intended and give highly comparable results, the main differences being how these handle regions of high intensities due to differences in how nearby distances are factored into summations; see, for example, the upper right corners of the plots.

Angular variants, as illustrated in Figure~\ref{fig:closeness_compare_ang}, demonstrate the same issue for \emph{Closeness}. \emph{Normalised Closeness} and \emph{NAIN} again flatten the distribution of values, though due to the use of geometric (angular) distances, these now have a tendency to emphasise rectilinear regions of the grid with the implication that the more densely interconnected portions of the network are less emphasised. The \emph{Improved} and \emph{Harmonic} formulations again behave as intended, once more yielding similar outputs to each other. Note that the \emph{Gravity} formulation is not applicable to geometric distances.

The above described similarities and differences between the measures are reflected in the correlation grid (Figure~\ref{fig:cent_corr_matrix_close}), where \emph{Normalised Closeness} differs markedly from the other forms and where \emph{Closeness} shows negative correlations, particularly when compared for smaller distances.

\begin{figure}[p]
  \centering
  \includegraphics[height=0.95\textheight, keepaspectratio]{../plots/closeness_compare.png}
  \caption{Comparative plots of shortest metric distance closeness centralities for a 1000m catchment.}\label{fig:closeness_compare}
\end{figure}

\begin{figure}[p]
  \centering
  \includegraphics[height=0.95\textheight, keepaspectratio]{../plots/closeness_compare_ang.png}
  \caption{Comparative plots of shortest geometric distance (angular) closeness centralities for a 1000m catchment. \emph{Gravity} is not applicable to geometric distances.}\label{fig:closeness_compare_ang}
\end{figure}

\begin{figure*}[htb]
  \centering
  \includegraphics[width=0.7\linewidth, keepaspectratio]{../plots/cent_corr_matrix_close.pdf}
  \caption{Correlation grid of different closeness centralities.}\label{fig:cent_corr_matrix_close}
\end{figure*}

\subsection{Comparisons to landuse accessibilities}

Figures \ref{fig:cent_lu_corrs}, \ref{fig:cent_lu_corrs_ang}, \ref{fig:cent_lu_corrs_length_wtd}, \ref{fig:cent_lu_corrs_length_wtd_ang} represent correlation grids comparing network centralities to landuse accessibilities. Each grid can be read as follows:
\begin{itemize}
  \item The $y$ axis labels correspond to a network centrality measure.
  \item The $x$ axis labels correspond to the distance used for the localised catchment (so-called radii or moving windows) at which the network centrality measures shown on the $y$ axis have been computed, ranging from $500m$ to $10km$.
  \item The correlations shown in the individual cells correspond to the Spearman Rank correlation for a given network centrality measure ($y$ axis) at a given distance ($x$ axis) correlated to the landuse theme for a given correlation matrix (indicated in the title).
\end{itemize}

\begin{figure}[htbp]
  \centering
  \includegraphics[width=0.75\textwidth, keepaspectratio]{../plots/cent_lu_corrs.pdf}
  \caption{Correlation grids comparing \emph{metric distance} network centralities to landuses.}\label{fig:cent_lu_corrs}
\end{figure}

\begin{figure}[htbp]
  \centering
  \includegraphics[width=0.75\textwidth, keepaspectratio]{../plots/cent_lu_corrs_ang.pdf}
  \caption{Correlation grids comparing \emph{geometric distance} angular network centralities to landuses.}\label{fig:cent_lu_corrs_ang}
\end{figure}

\begin{figure}[htbp]
  \centering
  \includegraphics[width=0.75\textwidth, keepaspectratio]{../plots/cent_lu_corrs_length_wtd.pdf}
  \caption{Correlation grids comparing length-weighted \emph{metric distance} network centralities to landuses.}\label{fig:cent_lu_corrs_length_wtd}
\end{figure}

\begin{figure}[htbp]
  \centering
  \includegraphics[width=0.75\textwidth, keepaspectratio]{../plots/cent_lu_corrs_length_wtd_ang.pdf}
  \caption{Correlation grids comparing length-weighted \emph{geometric distance} angular network centralities to landuses.}\label{fig:cent_lu_corrs_length_wtd_ang}
\end{figure}

The results are broadly summarised as follows:
\begin{itemize}
  \item The density measure represents a basic count of nodes, else of street lengths for streets inside the distance thresholds for the length-weighted case. Density shows strong associations for smaller distances in spite of its simplicity, potentially indicating that for smaller catchments the overriding issue is direct access to as much of the street network as possible. The \emph{Cycles} measure, a basic count of network cycles, performs similarly.
  \item The \emph{Farness} measure is positively associated and shows stronger associations for smaller distance thresholds for reasons likely similar to density, where simple direct access to the surrounding network affords greater access to landuses. For larger distances, it lags the more complex measures because it doesn't directly consider the effective closeness of the network as the network size expands. The implication of \emph{Farness} being positively associated is that its inverse, \emph{Closeness} correlates negatively. The normalised case of either measure is ineffective.
  \item \emph{NAIN}, where the numerator is raised to the power of 1.2, behaves more suitably than \emph{Normalised Closeness}, though lags the \emph{Improved} ($N^2$), \emph{Gravity}, and \emph{Harmonic Closeness} variants which show the most consistently strong associations. The latter two appear to slightly outperform \emph{Improved Closeness}, possibly because these factor distances directly for each summation instead of first summing and then averaging the distances prior to division. However, this trend is subtle and does not likely warrant favouring one over the other.
  \item In the context of Madrid, the betweenness variants are more weakly associated to landuses than closeness-like measures. This is to be expected given Madrid's high intensities of landuses in its walkable core, which follow a fractal or `space-filling' logic utilising all available street-frontages even where not directly adjacent to the most heavily walked streets. For similar reasons, Madrid generally shows slightly weaker associations for geometric (angular) distance measures. Note that these observations may be different for other contexts (e.g. London's high streets) or for associations against pedestrian volumes as opposed to landuse intensities.
  \item The street-length weighted variants and their accompanying continuous variants demonstrate similar behaviour, with generally slightly stronger associations compared to the unweighted versions.
\end{itemize}

\subsection{Comparisons to origin-destination trips data}

\begin{figure}[htbp]
  \centering
  \includegraphics[width=0.5\textwidth, keepaspectratio]{../plots/cent_ts_corrs.pdf}
  \caption{Correlation grids comparing average \emph{metric distance} network centralities to trip origins and destinations for travel survey zones.}\label{fig:cent_ts_corrs}
\end{figure}

\begin{figure}[htbp]
  \centering
  \includegraphics[width=0.5\textwidth, keepaspectratio]{../plots/cent_ts_corrs_ang.pdf}
  \caption{Correlation grids comparing average \emph{geometric distance} angular network centralities to trip origins and destinations for travel survey zones.}\label{fig:cent_ts_corrs_ang}
\end{figure}

Correlations of network centralities to origin destination travel survey data, shown in Figures~\ref{fig:cent_ts_corrs} and~\ref{fig:cent_ts_corrs_ang}, largely reflect the aforementioned patterns, with \emph{Closeness} and \emph{Normalised Closeness} again showing negative or negligible associations whereas the \emph{Improved}, \emph{Gravity}, and \emph{Harmonic Closeness} versions are more strongly associated.

\subsection{Limitations}

Although Madrid shows weaker associations for betweenness (compared to closeness) and \emph{geometric} angular distance measures (compared to  \emph{metric}) in the context of landuses, other contexts, such as studies based on detailed pedestrian counts or contexts such as London's neighbourhood high streets, would be expected to accentuate betweenness and \emph{geometric} versions of the measures more strongly.
