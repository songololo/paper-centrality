\section{Results}\label{data-analysis}

Following the methodology outlined in Section~\ref{empirical-methodology}, we present the results of the empirical analysis. A visual survey of the 1000m closeness centralities is plotted for \emph{metric} distance measures in Figure~\ref{fig:closeness_compare}, with \emph{geometric} distance (angular) measures shown in Figure~\ref{fig:closeness_compare_ang}. The correlation matrix for the discussed closeness centralities is shown in Figure~\ref{fig:cent_corr_matrix_close}.

\subsection{Visual Comparisons}

As illustrated in Figure~\ref{fig:closeness_compare}, \emph{Closeness} exhibits notable challenges due to its mathematical behaviour, which scales differently depending on the number of nodes under consideration. The iterative process of isolating catchments for localised calculations can therefore lead \emph{Closeness} to yield values opposite to its intended usage. \emph{Normalised Closeness} and \emph{NAIN} reduce the variance, but do not recover meaningful contrast across different street network intensities. The \emph{Improved}, \emph{Gravity}, and \emph{Harmonic} formulations behave as intended and give highly comparable results, the main differences being how these handle regions of high intensities due to differences in how nearby distances are factored into summations; see, for example, the upper right corners of the plots.

Angular variants, as illustrated in Figure~\ref{fig:closeness_compare_ang}, exhibit the same pattern for \emph{Closeness}. \emph{Normalised Closeness} and \emph{NAIN} again flatten the distribution of values, though due to the use of geometric (angular) distances, these now have a tendency to emphasise more orthogonal portions of the network with the implication that the more densely interconnected areas are less emphasised. The \emph{Improved} and \emph{Harmonic} formulations again behave as intended, once more yielding similar outputs to each other. Note that the \emph{Gravity} formulation is not applicable to geometric distances.

The above described similarities and differences between the measures are reflected in the correlation grid (Figure~\ref{fig:cent_corr_matrix_close}), where \emph{Normalised Closeness} differs markedly from the other forms and where \emph{Closeness} shows negative correlations, particularly when compared for smaller distances.

\subsection{Key Finding: Sign Reversal Across Formulations}

Before presenting detailed correlation grids, we highlight the central empirical finding. Block-bootstrap confidence intervals (accounting for spatial autocorrelation; see Section~\ref{sec:spatial_autocorr} and Table~\ref{tab:bootstrap_ci}) indicate a marked divergence in how different closeness formulations are associated with land-use intensity at localised scales:
\begin{itemize}
  \item \textbf{Closeness} is associated with a strong \emph{negative} correlation ($\rho \approx -0.70$, 95\% CI: $[-0.76, -0.64]$ at 500m), a sign reversal relative to the intended interpretation.
  \item \textbf{Normalised Closeness} ($N^1$) shows correlations near zero ($\rho \approx -0.11$, 95\% CI: $[-0.16, -0.07]$ at 500m), suggesting that simple normalisation does not address the underlying issue.
  \item \textbf{Improved Closeness} ($N^2$), \textbf{Gravity}, and \textbf{Harmonic Closeness} show consistently strong \emph{positive} correlations ($\rho \approx +0.67$, 95\% CI: $[+0.60, +0.72]$ at 500m for \emph{Improved Closeness}).
\end{itemize}
These confidence intervals account for the high spatial autocorrelation ($I \approx 0.79$--$0.84$) present in well-behaved closeness measures, which reduces the effective sample size from $N = 42{,}167$ segments to approximately 3,700 independent observations. The pattern persists across distance thresholds (500m--10km), for both metric and geometric distances, and for origin-destination trip data (Figures~\ref{fig:cent_ts_corrs},~\ref{fig:cent_ts_corrs_ang}). The following subsections present detailed correlation grids and the spatial autocorrelation analysis underpinning these estimates.

\begin{figure}[p]
  \centering
  \includegraphics[height=0.95\textheight, keepaspectratio]{plots/closeness_compare.png}
  \caption{Comparative plots of shortest metric distance closeness centralities for a 1000m catchment.}\label{fig:closeness_compare}
\end{figure}

\begin{figure}[p]
  \centering
  \includegraphics[height=0.95\textheight, keepaspectratio]{plots/closeness_compare_ang.png}
  \caption{Comparative plots of shortest geometric distance (angular) closeness centralities for a 1000m catchment. \emph{Gravity} is not applicable to geometric distances.}\label{fig:closeness_compare_ang}
\end{figure}

\begin{figure*}[htb]
  \centering
  \includegraphics[width=0.7\linewidth, keepaspectratio]{plots/cent_corr_matrix_close.pdf}
  \caption{Correlation grid of different closeness centralities.}\label{fig:cent_corr_matrix_close}
\end{figure*}

\subsection{Comparisons to Land-Use Accessibility}

Figures \ref{fig:cent_lu_corrs}, \ref{fig:cent_lu_corrs_ang}, \ref{fig:cent_lu_corrs_length_wtd}, \ref{fig:cent_lu_corrs_length_wtd_ang} represent correlation grids comparing network centralities to land-use accessibilities. Each grid can be read as follows:
\begin{itemize}
  \item The $y$ axis labels correspond to a network centrality measure.
  \item The $x$ axis labels correspond to the distance used for the localised catchment (so-called radii or moving windows) at which the network centrality measures shown on the $y$ axis have been computed, ranging from $500m$ to $10km$.
  \item The correlations shown in the individual cells correspond to the Spearman Rank correlation for a given network centrality measure ($y$ axis) at a given distance ($x$ axis) correlated to the land-use theme for a given correlation matrix (indicated in the title).
\end{itemize}

\begin{figure}[htbp]
  \centering
  \includegraphics[width=0.75\textwidth, keepaspectratio]{plots/cent_lu_corrs.pdf}
  \caption{Correlation grids comparing \emph{metric distance} network centralities to land uses.}\label{fig:cent_lu_corrs}
\end{figure}

\begin{figure}[htbp]
  \centering
  \includegraphics[width=0.75\textwidth, keepaspectratio]{plots/cent_lu_corrs_ang.pdf}
  \caption{Correlation grids comparing \emph{geometric distance} angular network centralities to land uses.}\label{fig:cent_lu_corrs_ang}
\end{figure}

\begin{figure}[htbp]
  \centering
  \includegraphics[width=0.75\textwidth, keepaspectratio]{plots/cent_lu_corrs_length_wtd.pdf}
  \caption{Correlation grids comparing length-weighted \emph{metric distance} network centralities to land uses.}\label{fig:cent_lu_corrs_length_wtd}
\end{figure}

\begin{figure}[htbp]
  \centering
  \includegraphics[width=0.75\textwidth, keepaspectratio]{plots/cent_lu_corrs_length_wtd_ang.pdf}
  \caption{Correlation grids comparing length-weighted \emph{geometric distance} angular network centralities to land uses.}\label{fig:cent_lu_corrs_length_wtd_ang}
\end{figure}

The results are broadly summarised as follows:
\begin{itemize}
  \item The density measure represents a basic count of nodes, else of street lengths for streets inside the distance thresholds for the length-weighted case. Density shows strong associations for smaller distances in spite of its simplicity, potentially indicating that for smaller catchments the overriding issue is direct access to as much of the street network as possible. The \emph{Cycles} measure, a basic count of network cycles, performs similarly.
  \item The \emph{Farness} measure is positively associated and shows stronger associations for smaller distance thresholds for reasons likely similar to density, where simple direct access to the surrounding network affords greater access to land uses. For larger distances, it lags the more complex measures because it doesn't directly consider the effective closeness of the network as the network size expands. The implication of \emph{Farness} being positively associated is that its inverse, \emph{Closeness} correlates negatively. The normalised case of either measure yields negligible associations.
  \item \emph{NAIN}, where the numerator is raised to the power of 1.2, behaves more suitably than \emph{Normalised Closeness}, though lags the \emph{Improved Closeness}, \emph{Gravity}, and \emph{Harmonic Closeness} variants which show the most consistently strong associations. The latter two appear to slightly outperform \emph{Improved Closeness}, possibly because these factor distances directly for each summation instead of first summing and then averaging the distances prior to division. However, this trend is subtle and does not likely warrant favouring one over the other.
  \item In the context of Madrid, the betweenness variants are more weakly associated to land uses than closeness-like measures. This is to be expected given Madrid's high intensities of land uses in its walkable core, which follow a fractal or `space-filling' logic utilising all available street-frontages even where not directly adjacent to the most heavily walked streets. For similar reasons, Madrid generally shows slightly weaker associations for geometric (angular) distance measures. Note that these observations may be different for other contexts (e.g. London's high streets) or for associations against pedestrian volumes as opposed to land-use intensities.
  \item The street-length weighted variants and their accompanying continuous variants exhibit similar behaviour, with generally slightly stronger associations compared to the unweighted versions.
\end{itemize}

\subsection{Comparisons to Origin-Destination Trip Data}

\begin{figure}[htbp]
  \centering
  \includegraphics[width=0.5\textwidth, keepaspectratio]{plots/cent_ts_corrs.pdf}
  \caption{Correlation grids comparing average \emph{metric distance} network centralities to trip origins and destinations for travel survey zones.}\label{fig:cent_ts_corrs}
\end{figure}

\begin{figure}[htbp]
  \centering
  \includegraphics[width=0.5\textwidth, keepaspectratio]{plots/cent_ts_corrs_ang.pdf}
  \caption{Correlation grids comparing average \emph{geometric distance} angular network centralities to trip origins and destinations for travel survey zones.}\label{fig:cent_ts_corrs_ang}
\end{figure}

Correlations of network centralities to origin-destination travel survey data, shown in Figures~\ref{fig:cent_ts_corrs} and~\ref{fig:cent_ts_corrs_ang}, largely reflect the aforementioned patterns, with \emph{Closeness} and \emph{Normalised Closeness} again showing negative or negligible associations whereas the \emph{Improved}, \emph{Gravity}, and \emph{Harmonic Closeness} versions are more strongly associated.

\subsection{Spatial Autocorrelation and Uncertainty Quantification}\label{sec:spatial_autocorr}

The spatial autocorrelation analysis (Table~\ref{tab:morans_i}) indicates a further divergence between closeness formulations. Measures that scale as anticipated --- \emph{Improved Closeness}, \emph{Gravity}, and \emph{Harmonic Closeness} --- exhibit high Moran's $I$ values ($I \approx 0.79$--$0.84$), consistent with the smooth spatial clustering typically associated with closeness centrality. Nearby street segments sharing similar network context have similarly pronounced centrality values under these formulations. In contrast, \emph{Closeness} and \emph{Normalised Closeness} exhibit lower spatial autocorrelation ($I \approx 0.18$--$0.2$ at 500m), consistent with less coherent spatial patterning.

To account for this spatial dependence when estimating uncertainty, we compute effective sample sizes and block-bootstrap confidence intervals. The effective sample size varies from approximately 3,700 for measures with high spatial coherence ($I \approx 0.84$) to 27,000 for \emph{Closeness} ($I \approx 0.21$), compared to the nominal $N = 42{,}167$ segments. The block-bootstrap procedure (Table~\ref{tab:bootstrap_ci}) uses spatially contiguous blocks to preserve autocorrelation structure, yielding conservative confidence intervals consistent with the sign-reversal pattern summarised in Section~\ref{data-analysis}: the intervals for \emph{Closeness}, \emph{Normalised Closeness}, and \emph{Improved Closeness} do not overlap, indicating that the observed divergence is statistically robust.

\subsection{Summary}

The empirical results are consistent with the theoretical hypothesis presented in Section~\ref{closeness_centrality}. \emph{Normalised Closeness}, despite being widely cited in the street network analysis literature, shows weak or negative associations with land-use accessibility and origin-destination trip volumes when computed for localised analyses. In contrast, \emph{Improved Closeness} and \emph{Harmonic Closeness} --- which scale appropriately across varying numbers of reachable nodes --- show consistently stronger associations. These findings support our contention that the divergence between cited formulations and those implemented in computational packages reflects a meaningful methodological distinction rather than notational variation.

