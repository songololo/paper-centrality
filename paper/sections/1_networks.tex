\section{Introduction}

Street network analysis enables the study of relationships between urban configuration and patterns of activity. Certain streets are livelier than others; land-uses cluster in particular locations; walkable configurations can become destinations in their own right \cite{Jacobs1961}. These emergent properties are difficult to anticipate for planned street networks without recourse to the incremental development typical of historical towns \cite{Alexander1967}. Network centrality measures make such properties tractable, relating the structure of street systems to characteristics such as accessibility and movement potential \cite{Hillier1984, Porta2009}. Space syntax theory, developed by Hillier and colleagues, integrated network measures with a social theory of space to analyse cities \cite{Hillier1984}, prompting wider adoption of street network analysis in urban design and planning.

Yet a documentation gap now threatens reproducibility. The closeness centrality formulas cited in published literature diverge from those implemented in widely used software packages. For global analysis on fixed graphs, this distinction may pass unnoticed. For localised analysis---where distance thresholds define varying subgraphs at each origin---the consequences are substantial. A researcher implementing the cited formula to replicate a study will obtain different results, potentially with reversed sign or negligible magnitude.

This matters because localised analysis is now standard practice. It mitigates edge effects, enables cross-city comparisons, and permits multi-scalar investigation of network properties \cite{Turner2007, van_nes_introduction_2021}. The shift from global to localised methods required adaptation of the underlying mathematics, but this transition is not well documented. Developers of packages such as \texttt{Depthmap} \cite{Hillier2012}, \texttt{Place Syntax Tool} \cite{stahle_place_2023}, and \texttt{cityseer} \cite{simons_cityseer_2023} adapted their closeness implementations accordingly, yet papers using these tools continue to cite formulations that behave differently under localised conditions.

The issue sits within a broader context of reproducibility challenges in street network analysis, including differences in network representation \cite{feng_accessibility_2022, Marshall2018}, algorithmic variations \cite{krenz_kimon_developments_2022}, and inadequate control for edge effects \cite{Gil2017}. The divergence between cited and implemented closeness formulations represents a specific, tractable instance of these challenges---one with clear mathematical explanation and practical resolution.

In this paper, we clarify the distinction between closeness formulations, explain why it matters for localised analysis, and demonstrate the behavioural differences empirically. We provide open code and data for reproducibility.

Our contribution is methodological, not substantive. We do not claim that any centrality formulation correctly captures urban accessibility, nor that correlations with land-use represent causal relationships. Rather, we demonstrate that formulations behave differently under localised analysis---a mathematical property---and use correlations with urban variables as a diagnostic tool to reveal this divergence. The theoretical argument is general; the specific correlation magnitudes serve only to show that the divergence has practical consequences.

\paragraph{What this paper adds:} We document a divergence between the closeness formulation cited in the street network analysis literature (\emph{Normalised Closeness}) and the variants implemented in computational packages (\emph{Improved Closeness}). We explain theoretically why this distinction matters for localised analysis: \emph{Closeness} exhibits sign reversal when subgraph sizes vary; \emph{Normalised Closeness} yields negligible associations; \emph{Improved} and \emph{Harmonic Closeness} scale as expected. We demonstrate these behaviours empirically and provide an openly reproducible workflow.

\paragraph{Research questions:} \textbf{RQ1:} Do different closeness formulations behave differently under localised street network analysis? \textbf{RQ2:} Are these mathematical differences empirically detectable when centrality measures are correlated with urban indicators? We hypothesise that \emph{Closeness} will exhibit sign reversal under localised analysis because it conflates subgraph size with average distance, and that \emph{Normalised Closeness} will not correct this. By contrast, \emph{Improved} and \emph{Harmonic Closeness} should scale appropriately. Section~\ref{closeness_centrality} develops this hypothesis formally; Sections~\ref{empirical-methodology}--\ref{data-analysis} demonstrate these differences empirically.
