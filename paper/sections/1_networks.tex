\section{Introduction}

Streets form connections between diverse activities and places. While their functional role is apparent, their emergent properties are less so. It can be observed that particular sidewalks are livelier than others, that certain landuses may have a proclivity for particular locations, or that walkable and vibrant streets can become public destinations in their own right \cite{Jacobs1961}. Yet, without recourse to the incremental and evolutionary development typical of historical towns and cities, it is not straightforward to anticipate these dynamics for planned street networks \cite{Alexander1967}. Even if the benefits of compact and walkable forms of urban development are widely reported, new developments continue to prioritise vehicular movement to more distant places at the expense of pedestrians and connectivity to local street networks \cite{transport_for_new_homes_what_2025}. This is where network analytic methods are particularly relevant: by enabling the comparison of emergent properties of street configurations, they make it feasible to explore the emergent properties of street systems and how these relate to potential activity and use.

Graph network methods have been widely utilised across numerous disciplines, variously adopting network analytical methods for purposes ranging from early centrality methods in the study of telecommunication networks \cite{Shimbel1953} to more recent advances in social network analysis \cite{Wasserman1994}. Street network analysis likewise emerged as a pivotal tool in urbanism, urban design, and urban planning. The utilisation of graph theories for spatial problems can be traced to 1736 when Euler proposed a graph-based solution to the Königsberg Seven Bridges problem \cite{euler_solutio_1741}. In the 1950s and 1960s, Froshaug compared hierarchies of street networks by mapping them as simplified graphs with street intersections as nodes, notably the ULM4 project in 1959 \cite{froshaug_visuelle_1959}. Space syntax theory, developed by Bill Hillier and colleagues in the late 1970s and early 1980s \cite{Hillier1984}, integrated a social theory of space with network measures to represent and analyse cities, prompting wider consideration on the usage of street network analysis for insights into urban design and planning.

Despite common foundations, the interpretation and application of centrality measures can vary across fields and traditions, leading to inconsistencies in terminology and implementation. Subtle differences in formulation or methodology can lead to significant changes in the mathematical behaviour of a measure, resulting in conflicting interpretations of the relevance of centralities to characteristics such as walkability or landuse accessibility. This may lead to spurious interpretations and widespread misattribution of results due to unacknowledged variations in the underlying metrics and methods. Previous authors have noted related challenges stemming from differences in model construction and representation \cite{feng_accessibility_2022, Marshall2018}, variations in algorithmic implementations and metric calculations \cite{krenz_kimon_developments_2022}, and the lack of buffering of network boundaries to control for edge effects \cite{Gil2017}. These inconsistencies present a significant barrier to the reproducibility, interpretation, and validity of results.

In the following analysis, we draw attention to the interpretation of closeness centralities in urban analytics, where we note a marked divergence between the formulations used by computational packages and the \emph{Normalised Closeness} variant widely cited in literature. We consequently ask whether these differences would lead to observable differences in behaviour in the context of contemporary localised network analysis when applied to real-world street networks. We provide open code and workflows based on open datasets for reproducibility.
