\subsection{Additional Figures and Tables}\label{additional-figures}

\begin{figure*}[htb]
  \centering
  \includegraphics[width=\linewidth, keepaspectratio]{plots/pca_corr_matrix.pdf}
  \caption{PCA loadings matrix.}\label{fig:pca_loadings}
\end{figure*}

\subsubsection{Descriptive Statistics Tables}

\textit{Note on closeness centrality values:} Closeness centrality measures exhibit small absolute values because they invert \emph{Farness} or \emph{Normalised Farness}, dividing node counts by cumulative distances in meters. For example, a 500m catchment with 100 nodes and 28,000m cumulative farness yields \emph{Closeness} $= 1 / 28{,}000 = 0.000036$ and \emph{Normalised Closeness} $= 100 / 28{,}000 = 0.00357$. These values are mathematically correct and do not affect statistical analyses, as Spearman correlations depend on rank order.

\begin{table}[htb]
  \centering
  \small
  \caption{Descriptive statistics for land use variables.}\label{tab:desc_landuse}
  \begin{tabular}{lrlllllll}
\toprule
 & N & Mean & Median & Q1 & Q3 & IQR & Min & Max \\
\midrule
pca\_1 & 42167 & -6.89e-04 & -0.439 & -3.177 & 2.809 & 5.986 & -5.900 & 10.000 \\
food\_bev\_200 & 42167 & 0.925 & 0.265 & 0.00e+00 & 1.134 & 1.134 & 0.00e+00 & 21 \\
retail\_200 & 42167 & 1.997 & 0.461 & 0.00e+00 & 2.328 & 2.328 & 0.00e+00 & 88 \\
\bottomrule
\end{tabular}

\end{table}

\begin{table}[htb]
  \centering
  \small
  \caption{Descriptive statistics for centrality measures.}\label{tab:desc_centrality}
  \begin{tabular}{lrlllllll}
\toprule
 & N & Mean & Median & Q1 & Q3 & IQR & Min & Max \\
\midrule
density\_500 & 42167 & 100 & 86 & 49 & 144 & 95 & 0.00e+00 & 367 \\
density\_1000 & 42167 & 352 & 330 & 192 & 496 & 304 & 0.00e+00 & 1092 \\
density\_2000 & 42167 & 1242 & 1199 & 796 & 1771 & 975 & 0.00e+00 & 2614 \\
density\_5000 & 42167 & 7078 & 7161 & 4790 & 9549 & 4759 & 0.00e+00 & 13191 \\
density\_10000 & 42167 & 25667 & 26803 & 21574 & 30862 & 9288 & 0.00e+00 & 36199 \\
far\_500 & 42167 & 32418 & 28121 & 15814 & 47338 & 31523 & 0.00e+00 & 122029 \\
far\_1000 & 42167 & 227738 & 216367 & 123304 & 316214 & 192910 & 0.00e+00 & 721591 \\
far\_2000 & 42167 & 1603791 & 1554475 & 978723 & 2329708 & 1350985 & 0.00e+00 & 3296895 \\
far\_5000 & 42167 & 23277712 & 23682352 & 15904454 & 30917964 & 15013510 & 0.00e+00 & 44062764 \\
far\_10000 & 42167 & 165974864 & 174153936 & 144394816 & 194249440 & 49854624 & 0.00e+00 & 223477152 \\
far\_norm\_500 & 41537 & 326 & 328 & 310 & 344 & 34 & 79 & 500 \\
far\_norm\_1000 & 42008 & 647 & 652 & 613 & 684 & 72 & 223 & 1000 \\
far\_norm\_2000 & 42147 & 1289 & 1296 & 1213 & 1378 & 165 & 532 & 1987 \\
far\_norm\_5000 & 42163 & 3295 & 3283 & 3168 & 3421 & 253 & 1422 & 4627 \\
far\_norm\_10000 & 42163 & 6552 & 6510 & 6204 & 6817 & 614 & 5290 & 9101 \\
closeness\_500 & 41537 & 8.05e-05 & 3.49e-05 & 2.10e-05 & 6.12e-05 & 4.02e-05 & 8.19e-06 & 6.97e-03 \\
closeness\_1000 & 42167 & 1.40e-05 & 4.59e-06 & 3.14e-06 & 8.01e-06 & 4.86e-06 & 0.00e+00 & 2.67e-03 \\
closeness\_2000 & 42147 & 1.95e-06 & 6.43e-07 & 4.29e-07 & 1.02e-06 & 5.92e-07 & 3.03e-07 & 5.77e-04 \\
closeness\_5000 & 42163 & 9.91e-08 & 4.22e-08 & 3.23e-08 & 6.29e-08 & 3.05e-08 & 2.27e-08 & 1.05e-04 \\
closeness\_10000 & 42163 & 7.50e-09 & 5.74e-09 & 5.15e-09 & 6.92e-09 & 1.78e-09 & 4.47e-09 & 1.60e-06 \\
close\_N1\_500 & 41537 & 3.10e-03 & 3.05e-03 & 2.91e-03 & 3.23e-03 & 3.17e-04 & 2.00e-03 & 1.27e-02 \\
close\_N1\_1000 & 42167 & 1.56e-03 & 1.53e-03 & 1.46e-03 & 1.63e-03 & 1.71e-04 & 0.00e+00 & 4.49e-03 \\
close\_N1\_2000 & 42147 & 7.85e-04 & 7.72e-04 & 7.26e-04 & 8.25e-04 & 9.90e-05 & 5.03e-04 & 1.88e-03 \\
close\_N1\_5000 & 42163 & 3.06e-04 & 3.05e-04 & 2.92e-04 & 3.16e-04 & 2.34e-05 & 2.16e-04 & 7.03e-04 \\
close\_N1\_10000 & 42163 & 1.53e-04 & 1.54e-04 & 1.47e-04 & 1.61e-04 & 1.45e-05 & 1.10e-04 & 1.89e-04 \\
close\_N1.2\_500 & 41537 & 7.48e-03 & 7.59e-03 & 6.72e-03 & 8.41e-03 & 1.69e-03 & 2.00e-03 & 1.46e-02 \\
close\_N1.2\_1000 & 42167 & 4.85e-03 & 4.92e-03 & 4.38e-03 & 5.47e-03 & 1.09e-03 & 0.00e+00 & 1.01e-02 \\
close\_N1.2\_2000 & 42147 & 3.16e-03 & 3.21e-03 & 2.90e-03 & 3.52e-03 & 6.23e-04 & 5.03e-04 & 4.98e-03 \\
close\_N1.2\_5000 & 42163 & 1.76e-03 & 1.81e-03 & 1.65e-03 & 1.94e-03 & 2.91e-04 & 3.05e-04 & 2.15e-03 \\
close\_N1.2\_10000 & 42163 & 1.16e-03 & 1.18e-03 & 1.09e-03 & 1.27e-03 & 1.80e-04 & 3.14e-04 & 1.37e-03 \\
close\_N2\_500 & 41537 & 0.312 & 0.272 & 0.158 & 0.447 & 0.289 & 2.00e-03 & 1.172 \\
close\_N2\_1000 & 42167 & 0.549 & 0.509 & 0.298 & 0.777 & 0.479 & 0.00e+00 & 1.728 \\
close\_N2\_2000 & 42147 & 0.969 & 0.946 & 0.609 & 1.336 & 0.726 & 5.03e-04 & 2.264 \\
close\_N2\_5000 & 42163 & 2.159 & 2.180 & 1.451 & 2.964 & 1.513 & 9.25e-04 & 3.954 \\
close\_N2\_10000 & 42163 & 3.982 & 4.137 & 3.208 & 4.905 & 1.697 & 1.47e-02 & 6.007 \\
harmonic\_500 & 42167 & 0.407 & 0.354 & 0.201 & 0.583 & 0.382 & 0.00e+00 & 1.721 \\
harmonic\_1000 & 42167 & 0.747 & 0.685 & 0.408 & 1.059 & 0.651 & 0.00e+00 & 2.532 \\
harmonic\_2000 & 42167 & 1.343 & 1.315 & 0.834 & 1.838 & 1.004 & 0.00e+00 & 3.564 \\
harmonic\_5000 & 42167 & 3.011 & 3.025 & 2.071 & 4.110 & 2.038 & 0.00e+00 & 6.155 \\
harmonic\_10000 & 42167 & 5.524 & 5.730 & 4.291 & 6.952 & 2.661 & 0.00e+00 & 9.433 \\
gravity\_500 & 42167 & 12 & 10 & 5.901 & 17 & 11 & 0.00e+00 & 47 \\
gravity\_1000 & 42167 & 44 & 40 & 23 & 63 & 40 & 0.00e+00 & 154 \\
gravity\_2000 & 42167 & 157 & 152 & 93 & 216 & 124 & 0.00e+00 & 436 \\
gravity\_5000 & 42167 & 856 & 840 & 569 & 1189 & 621 & 0.00e+00 & 1688 \\
gravity\_10000 & 42167 & 3169 & 3252 & 2312 & 4135 & 1823 & 0.00e+00 & 5272 \\
\bottomrule
\end{tabular}

\end{table}

\begin{table}[htb]
  \centering
  \small
  \caption{Descriptive statistics for length-weighted centrality measures.}\label{tab:desc_centrality_lw}
  \begin{tabular}{lrlllllll}
\toprule
 & N & Mean & Median & Q1 & Q3 & IQR & Min & Max \\
\midrule
lw\_density\_500 & 42167 & 7348 & 7184 & 4856 & 9915 & 5059 & 0.00e+00 & 18311 \\
lw\_density\_1000 & 42167 & 28119 & 28248 & 19274 & 36788 & 17514 & 0.00e+00 & 64442 \\
lw\_density\_2000 & 42167 & 108908 & 108452 & 79721 & 146356 & 66635 & 0.00e+00 & 195076 \\
lw\_density\_5000 & 42167 & 690422 & 713402 & 531238 & 875962 & 344724 & 0.00e+00 & 1141834 \\
lw\_density\_10000 & 42167 & 2659032 & 2759148 & 2332424 & 3079680 & 747256 & 0.00e+00 & 3581994 \\
lw\_far\_500 & 42167 & 2446879 & 2390021 & 1592962 & 3301552 & 1708590 & 0.00e+00 & 6266275 \\
lw\_far\_1000 & 42167 & 18561794 & 18738480 & 12520390 & 24216358 & 11695968 & 0.00e+00 & 43358996 \\
lw\_far\_2000 & 42167 & 144247104 & 144698080 & 104013600 & 193948736 & 89935136 & 0.00e+00 & 257134432 \\
lw\_far\_5000 & 42167 & 2311166464 & 2394943744 & 1786027776 & 2902958592 & 1116930816 & 0.00e+00 & 3842761216 \\
lw\_far\_10000 & 42167 & 17492131840 & 18184536064 & 15647161344 & 19928051712 & 4280890368 & 0.00e+00 & 22876850176 \\
lw\_far\_norm\_500 & 41537 & 334 & 334 & 321 & 346 & 24 & 79 & 500 \\
lw\_far\_norm\_1000 & 42008 & 661 & 662 & 635 & 685 & 49 & 345 & 1000 \\
lw\_far\_norm\_2000 & 42147 & 1322 & 1324 & 1274 & 1375 & 101 & 764 & 1987 \\
lw\_far\_norm\_5000 & 42163 & 3356 & 3342 & 3277 & 3434 & 157 & 2420 & 4625 \\
lw\_far\_norm\_10000 & 42163 & 6626 & 6579 & 6392 & 6821 & 430 & 6154 & 8590 \\
lw\_closeness\_500 & 41537 & 6.98e-07 & 4.14e-07 & 3.01e-07 & 6.13e-07 & 3.11e-07 & 1.60e-07 & 4.33e-04 \\
lw\_closeness\_1000 & 42008 & 9.47e-08 & 5.32e-08 & 4.12e-08 & 7.94e-08 & 3.81e-08 & 2.31e-08 & 1.99e-04 \\
lw\_closeness\_2000 & 42147 & 1.06e-08 & 6.91e-09 & 5.16e-09 & 9.61e-09 & 4.45e-09 & 3.89e-09 & 1.42e-05 \\
lw\_closeness\_5000 & 42163 & 5.55e-10 & 4.18e-10 & 3.44e-10 & 5.60e-10 & 2.15e-10 & 2.60e-10 & 7.09e-08 \\
lw\_closeness\_10000 & 42163 & 6.26e-11 & 5.50e-11 & 5.02e-11 & 6.39e-11 & 1.37e-11 & 4.37e-11 & 2.27e-09 \\
lw\_close\_N1\_500 & 41537 & 3.01e-03 & 3.00e-03 & 2.89e-03 & 3.11e-03 & 2.19e-04 & 2.00e-03 & 1.27e-02 \\
lw\_close\_N1\_1000 & 42008 & 1.52e-03 & 1.51e-03 & 1.46e-03 & 1.57e-03 & 1.13e-04 & 1.00e-03 & 2.90e-03 \\
lw\_close\_N1\_2000 & 42147 & 7.60e-04 & 7.55e-04 & 7.27e-04 & 7.85e-04 & 5.79e-05 & 5.03e-04 & 1.31e-03 \\
lw\_close\_N1\_5000 & 42163 & 2.99e-04 & 2.99e-04 & 2.91e-04 & 3.05e-04 & 1.40e-05 & 2.16e-04 & 4.13e-04 \\
lw\_close\_N1\_10000 & 42163 & 1.51e-04 & 1.52e-04 & 1.47e-04 & 1.56e-04 & 9.85e-06 & 1.16e-04 & 1.62e-04 \\
lw\_close\_N1.2\_500 & 41537 & 1.75e-02 & 1.79e-02 & 1.65e-02 & 1.91e-02 & 2.65e-03 & 2.99e-03 & 3.71e-02 \\
lw\_close\_N1.2\_1000 & 42008 & 1.16e-02 & 1.18e-02 & 1.09e-02 & 1.26e-02 & 1.70e-03 & 1.41e-03 & 1.67e-02 \\
lw\_close\_N1.2\_2000 & 42147 & 7.59e-03 & 7.74e-03 & 7.16e-03 & 8.19e-03 & 1.03e-03 & 1.03e-03 & 9.17e-03 \\
lw\_close\_N1.2\_5000 & 42163 & 4.34e-03 & 4.43e-03 & 4.14e-03 & 4.69e-03 & 5.49e-04 & 1.24e-03 & 4.97e-03 \\
lw\_close\_N1.2\_10000 & 42163 & 2.90e-03 & 2.94e-03 & 2.78e-03 & 3.10e-03 & 3.13e-04 & 1.14e-03 & 3.28e-03 \\
lw\_close\_N2\_500 & 41537 & 22 & 22 & 15 & 30 & 15 & 1.47e-02 & 55 \\
lw\_close\_N2\_1000 & 42008 & 43 & 43 & 29 & 56 & 27 & 5.19e-03 & 96 \\
lw\_close\_N2\_2000 & 42147 & 83 & 83 & 60 & 109 & 49 & 1.78e-02 & 155 \\
lw\_close\_N2\_5000 & 42163 & 207 & 212 & 157 & 265 & 108 & 1.360 & 340 \\
lw\_close\_N2\_10000 & 42163 & 405 & 420 & 346 & 479 & 133 & 7.206 & 569 \\
lw\_harmonic\_500 & 42167 & 28 & 27 & 18 & 38 & 19 & 0.00e+00 & 74 \\
lw\_harmonic\_1000 & 42167 & 56 & 55 & 38 & 73 & 35 & 0.00e+00 & 132 \\
lw\_harmonic\_2000 & 42167 & 109 & 111 & 80 & 142 & 62 & 0.00e+00 & 219 \\
lw\_harmonic\_5000 & 42167 & 275 & 278 & 211 & 355 & 144 & 0.00e+00 & 463 \\
lw\_harmonic\_10000 & 42167 & 540 & 557 & 453 & 647 & 194 & 0.00e+00 & 799 \\
lw\_gravity\_500 & 42167 & 818 & 803 & 532 & 1112 & 580 & 0.00e+00 & 2166 \\
lw\_gravity\_1000 & 42167 & 3273 & 3231 & 2231 & 4358 & 2127 & 0.00e+00 & 7872 \\
lw\_gravity\_2000 & 42167 & 12701 & 12858 & 9125 & 16376 & 7251 & 0.00e+00 & 26269 \\
lw\_gravity\_5000 & 42167 & 77914 & 78117 & 59244 & 102658 & 43414 & 0.00e+00 & 131299 \\
lw\_gravity\_10000 & 42167 & 309890 & 319353 & 251567 & 383456 & 131889 & 0.00e+00 & 465867 \\
\bottomrule
\end{tabular}

\end{table}

\begin{table}[htb]
  \centering
  \small
  \caption{Descriptive statistics for angular centrality measures.}\label{tab:desc_centrality_ang}
  \begin{tabular}{lrlllllll}
\toprule
 & N & Mean & Median & Q1 & Q3 & IQR & Min & Max \\
\midrule
far\_500\_ang & 42167 & 260 & 209 & 112 & 375 & 263 & 0.00e+00 & 1592 \\
far\_1000\_ang & 42167 & 1300 & 1144 & 627 & 1836 & 1208 & 0.00e+00 & 7665 \\
far\_2000\_ang & 42167 & 6462 & 6209 & 3887 & 8985 & 5098 & 0.00e+00 & 27812 \\
far\_5000\_ang & 42167 & 51751 & 53623 & 30026 & 70395 & 40369 & 0.00e+00 & 165195 \\
far\_10000\_ang & 42167 & 355383 & 372069 & 282992 & 429857 & 146865 & 0.00e+00 & 802588 \\
far\_norm\_500\_ang & 41537 & 3.054 & 2.935 & 2.393 & 3.584 & 1.192 & 9.03e-03 & 12 \\
far\_norm\_1000\_ang & 42008 & 4.665 & 4.555 & 3.794 & 5.393 & 1.600 & 5.06e-02 & 14 \\
far\_norm\_2000\_ang & 42147 & 7.248 & 7.069 & 6.121 & 8.148 & 2.027 & 0.384 & 21 \\
far\_norm\_5000\_ang & 42163 & 12 & 12 & 11 & 14 & 2.625 & 2.412 & 37 \\
far\_norm\_10000\_ang & 42163 & 19 & 19 & 17 & 21 & 3.574 & 9.871 & 48 \\
closeness\_500\_ang & 41537 & 3.18e-02 & 4.71e-03 & 2.64e-03 & 8.64e-03 & 6.00e-03 & 6.28e-04 & 111 \\
closeness\_1000\_ang & 42167 & 4.85e-03 & 8.67e-04 & 5.41e-04 & 1.58e-03 & 1.04e-03 & 0.00e+00 & 20 \\
closeness\_2000\_ang & 42147 & 7.18e-04 & 1.61e-04 & 1.11e-04 & 2.57e-04 & 1.46e-04 & 3.60e-05 & 2.603 \\
closeness\_5000\_ang & 42163 & 4.67e-05 & 1.86e-05 & 1.42e-05 & 3.33e-05 & 1.91e-05 & 6.05e-06 & 8.80e-02 \\
closeness\_10000\_ang & 42163 & 3.64e-06 & 2.69e-06 & 2.33e-06 & 3.53e-06 & 1.21e-06 & 1.25e-06 & 5.11e-04 \\
close\_N1\_500\_ang & 41537 & 0.385 & 0.341 & 0.279 & 0.418 & 0.139 & 8.33e-02 & 111 \\
close\_N1\_1000\_ang & 42167 & 0.235 & 0.219 & 0.185 & 0.263 & 7.84e-02 & 0.00e+00 & 20 \\
close\_N1\_2000\_ang & 42147 & 0.147 & 0.141 & 0.123 & 0.163 & 4.06e-02 & 4.86e-02 & 2.603 \\
close\_N1\_5000\_ang & 42163 & 8.36e-02 & 8.29e-02 & 7.40e-02 & 9.19e-02 & 1.79e-02 & 2.67e-02 & 0.415 \\
close\_N1\_10000\_ang & 42163 & 5.33e-02 & 5.28e-02 & 4.81e-02 & 5.80e-02 & 9.97e-03 & 2.06e-02 & 0.101 \\
close\_N1.2\_500\_ang & 41537 & 0.851 & 0.817 & 0.654 & 0.986 & 0.332 & 0.147 & 111 \\
close\_N1.2\_1000\_ang & 42167 & 0.677 & 0.670 & 0.548 & 0.794 & 0.245 & 0.00e+00 & 20 \\
close\_N1.2\_2000\_ang & 42147 & 0.545 & 0.545 & 0.458 & 0.629 & 0.171 & 8.87e-02 & 2.603 \\
close\_N1.2\_5000\_ang & 42163 & 0.433 & 0.436 & 0.371 & 0.497 & 0.126 & 9.17e-02 & 0.789 \\
close\_N1.2\_10000\_ang & 42163 & 0.379 & 0.380 & 0.338 & 0.424 & 8.69e-02 & 8.04e-02 & 0.665 \\
close\_N2\_500\_ang & 41537 & 28 & 26 & 14 & 40 & 25 & 0.212 & 111 \\
close\_N2\_1000\_ang & 42167 & 60 & 57 & 32 & 83 & 51 & 0.00e+00 & 231 \\
close\_N2\_2000\_ang & 42147 & 127 & 116 & 72 & 177 & 105 & 9.12e-02 & 378 \\
close\_N2\_5000\_ang & 42163 & 358 & 343 & 187 & 510 & 322 & 0.382 & 933 \\
close\_N2\_10000\_ang & 42163 & 1024 & 1023 & 775 & 1317 & 542 & 3.257 & 2332 \\
harmonic\_500\_ang & 42167 & 24 & 22 & 12 & 33 & 21 & 0.00e+00 & 90 \\
harmonic\_1000\_ang & 42167 & 56 & 53 & 31 & 78 & 47 & 0.00e+00 & 208 \\
harmonic\_2000\_ang & 42167 & 128 & 120 & 75 & 177 & 102 & 0.00e+00 & 369 \\
harmonic\_5000\_ang & 42167 & 380 & 366 & 211 & 533 & 322 & 0.00e+00 & 951 \\
harmonic\_10000\_ang & 42167 & 1122 & 1123 & 835 & 1434 & 598 & 0.00e+00 & 2542 \\
\bottomrule
\end{tabular}

\end{table}

\begin{table}[htb]
  \centering
  \small
  \caption{Descriptive statistics for length-weighted angular centrality measures.}\label{tab:desc_centrality_lw_ang}
  \begin{tabular}{lrlllllll}
\toprule
 & N & Mean & Median & Q1 & Q3 & IQR & Min & Max \\
\midrule
lw\_far\_500\_ang & 42167 & 18826 & 17213 & 10848 & 25695 & 14847 & 0.00e+00 & 84449 \\
lw\_far\_1000\_ang & 42167 & 101824 & 97161 & 63036 & 136294 & 73258 & 0.00e+00 & 470177 \\
lw\_far\_2000\_ang & 42167 & 555098 & 555054 & 384851 & 733383 & 348533 & 0.00e+00 & 1902248 \\
lw\_far\_5000\_ang & 42167 & 4896063 & 5050740 & 3259989 & 6327224 & 3067235 & 0.00e+00 & 14512521 \\
lw\_far\_10000\_ang & 42167 & 35232836 & 36230860 & 29151620 & 41564344 & 12412724 & 0.00e+00 & 76790216 \\
lw\_far\_norm\_500\_ang & 41537 & 3.031 & 2.922 & 2.358 & 3.584 & 1.226 & 9.03e-03 & 12 \\
lw\_far\_norm\_1000\_ang & 42008 & 4.631 & 4.545 & 3.761 & 5.384 & 1.623 & 5.06e-02 & 14 \\
lw\_far\_norm\_2000\_ang & 42147 & 7.198 & 7.079 & 6.072 & 8.146 & 2.074 & 0.384 & 20 \\
lw\_far\_norm\_5000\_ang & 42163 & 12 & 12 & 11 & 13 & 2.699 & 2.341 & 32 \\
lw\_far\_norm\_10000\_ang & 42163 & 19 & 18 & 17 & 20 & 3.602 & 10 & 47 \\
lw\_closeness\_500\_ang & 41537 & 2.38e-04 & 5.73e-05 & 3.87e-05 & 8.98e-05 & 5.11e-05 & 1.18e-05 & 0.773 \\
lw\_closeness\_1000\_ang & 42008 & 2.93e-05 & 1.03e-05 & 7.33e-06 & 1.58e-05 & 8.44e-06 & 2.13e-06 & 9.80e-02 \\
lw\_closeness\_2000\_ang & 42147 & 3.33e-06 & 1.80e-06 & 1.36e-06 & 2.60e-06 & 1.23e-06 & 5.26e-07 & 5.32e-03 \\
lw\_closeness\_5000\_ang & 42163 & 2.80e-07 & 1.98e-07 & 1.58e-07 & 3.07e-07 & 1.49e-07 & 6.89e-08 & 5.61e-05 \\
lw\_closeness\_10000\_ang & 42163 & 3.20e-08 & 2.76e-08 & 2.41e-08 & 3.43e-08 & 1.02e-08 & 1.30e-08 & 1.69e-06 \\
lw\_close\_N1\_500\_ang & 41537 & 0.391 & 0.342 & 0.279 & 0.424 & 0.145 & 8.58e-02 & 111 \\
lw\_close\_N1\_1000\_ang & 42008 & 0.239 & 0.220 & 0.186 & 0.266 & 8.02e-02 & 7.20e-02 & 20 \\
lw\_close\_N1\_2000\_ang & 42147 & 0.148 & 0.141 & 0.123 & 0.165 & 4.19e-02 & 5.08e-02 & 2.603 \\
lw\_close\_N1\_5000\_ang & 42163 & 8.57e-02 & 8.40e-02 & 7.52e-02 & 9.44e-02 & 1.92e-02 & 3.08e-02 & 0.427 \\
lw\_close\_N1\_10000\_ang & 42163 & 5.54e-02 & 5.46e-02 & 4.96e-02 & 6.04e-02 & 1.08e-02 & 2.15e-02 & 9.56e-02 \\
lw\_close\_N1.2\_500\_ang & 41537 & 2.113 & 1.964 & 1.582 & 2.405 & 0.823 & 0.398 & 299 \\
lw\_close\_N1.2\_1000\_ang & 42008 & 1.694 & 1.629 & 1.355 & 1.943 & 0.588 & 0.294 & 58 \\
lw\_close\_N1.2\_2000\_ang & 42147 & 1.368 & 1.345 & 1.141 & 1.561 & 0.420 & 0.211 & 8.980 \\
lw\_close\_N1.2\_5000\_ang & 42163 & 1.118 & 1.116 & 0.958 & 1.266 & 0.308 & 0.292 & 3.208 \\
lw\_close\_N1.2\_10000\_ang & 42163 & 0.998 & 0.989 & 0.881 & 1.105 & 0.224 & 0.309 & 1.749 \\
lw\_close\_N2\_500\_ang & 41537 & 2185 & 2114 & 1398 & 2879 & 1481 & 2.476 & 33649 \\
lw\_close\_N2\_1000\_ang & 42008 & 4919 & 4738 & 3209 & 6382 & 3173 & 2.548 & 17205 \\
lw\_close\_N2\_2000\_ang & 42147 & 11114 & 10363 & 7302 & 14569 & 7267 & 3.655 & 31067 \\
lw\_close\_N2\_5000\_ang & 42163 & 35165 & 33939 & 22710 & 47038 & 24328 & 558 & 91400 \\
lw\_close\_N2\_10000\_ang & 42163 & 108127 & 107119 & 85866 & 133944 & 48078 & 2525 & 239304 \\
lw\_harmonic\_500\_ang & 42167 & 1775 & 1756 & 1173 & 2359 & 1186 & 0.00e+00 & 5308 \\
lw\_harmonic\_1000\_ang & 42167 & 4547 & 4446 & 3060 & 5897 & 2836 & 0.00e+00 & 14975 \\
lw\_harmonic\_2000\_ang & 42167 & 11221 & 10653 & 7638 & 14521 & 6883 & 0.00e+00 & 31891 \\
lw\_harmonic\_5000\_ang & 42167 & 37311 & 36086 & 24622 & 49110 & 24487 & 0.00e+00 & 98912 \\
lw\_harmonic\_10000\_ang & 42167 & 119072 & 117615 & 93104 & 146571 & 53467 & 0.00e+00 & 272356 \\
\bottomrule
\end{tabular}

\end{table}

\subsubsection{Zone-Level Descriptive Statistics}

Tables~\ref{tab:desc_zone_outcomes}--\ref{tab:desc_zone_centrality_ang} present zone-level descriptive statistics, with segment centralities averaged to origin-destination travel zones. Trip counts represent trips originating from or destined to each zone, normalised by zone area (km\textsuperscript{2}).

\begin{table}[htb]
  \centering
  \small
  \caption{Descriptive statistics for zone-level outcome variables (trip counts and trips per area).}\label{tab:desc_zone_outcomes}
  \begin{tabular}{lrlllllll}
\toprule
 & N & Mean & Median & Q1 & Q3 & IQR & Min & Max \\
\midrule
origin\_count & 625 & 114 & 106 & 64 & 155 & 91 & 1.000 & 444 \\
dest\_count & 625 & 114 & 106 & 64 & 156 & 92 & 1.000 & 458 \\
origin\_by\_area & 625 & 396 & 333 & 126 & 528 & 402 & 7.08e-02 & 5318 \\
dest\_by\_area & 625 & 400 & 331 & 131 & 520 & 389 & 3.54e-02 & 5383 \\
\bottomrule
\end{tabular}

\end{table}

\begin{table}[htb]
  \centering
  \small
  \caption{Descriptive statistics for zone-averaged centrality measures.}\label{tab:desc_zone_centrality}
  \begin{tabular}{lrlllllll}
\toprule
 & N & Mean & Median & Q1 & Q3 & IQR & Min & Max \\
\midrule
density\_500 & 625 & 82 & 70 & 41 & 113 & 71 & 0.333 & 303 \\
density\_1000 & 625 & 316 & 294 & 179 & 437 & 258 & 1.667 & 1009 \\
density\_2000 & 625 & 1235 & 1223 & 799 & 1740 & 941 & 4.333 & 2547 \\
density\_5000 & 625 & 7441 & 7682 & 5257 & 9798 & 4541 & 35 & 12708 \\
density\_10000 & 625 & 26321 & 27463 & 22581 & 31338 & 8757 & 632 & 36024 \\
far\_500 & 625 & 27013 & 23059 & 13407 & 36681 & 23274 & 81 & 100420 \\
far\_1000 & 625 & 208657 & 195084 & 117325 & 285472 & 168147 & 1205 & 637461 \\
far\_2000 & 625 & 1637380 & 1608266 & 1032858 & 2330804 & 1297946 & 5381 & 3166530 \\
far\_5000 & 625 & 24622024 & 25368328 & 17765866 & 32018626 & 14252760 & 118084 & 42565468 \\
far\_10000 & 625 & 169181600 & 177521104 & 149557632 & 196353424 & 46795792 & 5506382 & 221925904 \\
far\_norm\_500 & 625 & 333 & 332 & 324 & 340 & 16 & 228 & 467 \\
far\_norm\_1000 & 625 & 668 & 670 & 644 & 695 & 51 & 488 & 846 \\
far\_norm\_2000 & 625 & 1338 & 1347 & 1270 & 1409 & 139 & 895 & 1739 \\
far\_norm\_5000 & 625 & 3327 & 3309 & 3197 & 3453 & 256 & 2488 & 4063 \\
far\_norm\_10000 & 625 & 6495 & 6444 & 6195 & 6730 & 536 & 5903 & 8175 \\
closeness\_500 & 625 & 1.33e-04 & 5.29e-05 & 3.06e-05 & 1.21e-04 & 9.06e-05 & 1.01e-05 & 4.13e-03 \\
closeness\_1000 & 625 & 1.84e-05 & 5.96e-06 & 3.71e-06 & 1.28e-05 & 9.10e-06 & 1.57e-06 & 5.56e-04 \\
closeness\_2000 & 625 & 1.89e-06 & 6.52e-07 & 4.37e-07 & 1.09e-06 & 6.57e-07 & 3.16e-07 & 1.26e-04 \\
closeness\_5000 & 625 & 7.46e-08 & 3.96e-08 & 3.13e-08 & 5.71e-08 & 2.58e-08 & 2.35e-08 & 5.67e-06 \\
closeness\_10000 & 625 & 7.21e-09 & 5.64e-09 & 5.09e-09 & 6.72e-09 & 1.62e-09 & 4.51e-09 & 3.38e-07 \\
close\_N1\_500 & 625 & 3.05e-03 & 3.03e-03 & 2.95e-03 & 3.11e-03 & 1.57e-04 & 2.14e-03 & 5.09e-03 \\
close\_N1\_1000 & 625 & 1.51e-03 & 1.50e-03 & 1.44e-03 & 1.56e-03 & 1.15e-04 & 1.20e-03 & 2.13e-03 \\
close\_N1\_2000 & 625 & 7.55e-04 & 7.45e-04 & 7.11e-04 & 7.90e-04 & 7.81e-05 & 5.76e-04 & 1.15e-03 \\
close\_N1\_5000 & 625 & 3.02e-04 & 3.02e-04 & 2.90e-04 & 3.13e-04 & 2.30e-05 & 2.47e-04 & 4.13e-04 \\
close\_N1\_10000 & 625 & 1.54e-04 & 1.55e-04 & 1.49e-04 & 1.61e-04 & 1.28e-05 & 1.23e-04 & 1.70e-04 \\
close\_N1.2\_500 & 625 & 6.98e-03 & 7.03e-03 & 6.32e-03 & 7.76e-03 & 1.45e-03 & 2.14e-03 & 9.68e-03 \\
close\_N1.2\_1000 & 625 & 4.58e-03 & 4.62e-03 & 4.13e-03 & 5.13e-03 & 9.98e-04 & 1.65e-03 & 6.65e-03 \\
close\_N1.2\_2000 & 625 & 3.04e-03 & 3.08e-03 & 2.79e-03 & 3.38e-03 & 5.90e-04 & 1.18e-03 & 4.14e-03 \\
close\_N1.2\_5000 & 625 & 1.77e-03 & 1.82e-03 & 1.66e-03 & 1.94e-03 & 2.75e-04 & 6.46e-04 & 2.11e-03 \\
close\_N1.2\_10000 & 625 & 1.18e-03 & 1.20e-03 & 1.11e-03 & 1.27e-03 & 1.64e-04 & 4.45e-04 & 1.36e-03 \\
close\_N2\_500 & 625 & 0.253 & 0.215 & 0.132 & 0.341 & 0.209 & 2.14e-03 & 0.917 \\
close\_N2\_1000 & 625 & 0.482 & 0.435 & 0.266 & 0.656 & 0.390 & 3.52e-03 & 1.600 \\
close\_N2\_2000 & 625 & 0.938 & 0.928 & 0.590 & 1.274 & 0.684 & 5.24e-03 & 2.194 \\
close\_N2\_5000 & 625 & 2.256 & 2.313 & 1.576 & 3.029 & 1.453 & 1.53e-02 & 3.794 \\
close\_N2\_10000 & 625 & 4.108 & 4.253 & 3.387 & 5.008 & 1.621 & 0.109 & 5.920 \\
harmonic\_500 & 625 & 0.328 & 0.281 & 0.173 & 0.453 & 0.280 & 8.18e-04 & 1.189 \\
harmonic\_1000 & 625 & 0.640 & 0.575 & 0.360 & 0.853 & 0.493 & 2.99e-03 & 2.153 \\
harmonic\_2000 & 625 & 1.253 & 1.232 & 0.791 & 1.693 & 0.903 & 4.75e-03 & 3.212 \\
harmonic\_5000 & 625 & 3.032 & 3.066 & 2.195 & 4.058 & 1.863 & 1.33e-02 & 5.840 \\
harmonic\_10000 & 625 & 5.591 & 5.771 & 4.549 & 6.953 & 2.404 & 8.04e-02 & 9.100 \\
gravity\_500 & 625 & 9.658 & 8.153 & 5.057 & 13 & 8.306 & 1.28e-02 & 35 \\
gravity\_1000 & 625 & 37 & 33 & 20 & 50 & 30 & 0.178 & 130 \\
gravity\_2000 & 625 & 143 & 138 & 88 & 193 & 105 & 0.594 & 410 \\
gravity\_5000 & 625 & 865 & 861 & 611 & 1187 & 575 & 3.634 & 1658 \\
gravity\_10000 & 625 & 3272 & 3372 & 2483 & 4183 & 1700 & 28 & 5179 \\
cycles\_500 & 625 & 89 & 72 & 38 & 125 & 87 & 0.00e+00 & 346 \\
cycles\_1000 & 625 & 374 & 333 & 192 & 523 & 331 & 0.333 & 1234 \\
cycles\_2000 & 625 & 1522 & 1439 & 890 & 2226 & 1335 & 2.333 & 3367 \\
cycles\_5000 & 625 & 9357 & 9469 & 6286 & 12628 & 6342 & 21 & 16617 \\
cycles\_10000 & 625 & 32896 & 34557 & 28164 & 39556 & 11392 & 705 & 45055 \\
betw\_500 & 625 & 168 & 106 & 48 & 239 & 191 & 0.00e+00 & 1096 \\
betw\_1000 & 625 & 1338 & 981 & 458 & 1862 & 1404 & 0.00e+00 & 7960 \\
betw\_2000 & 625 & 10448 & 9034 & 4600 & 14875 & 10275 & 0.00e+00 & 39782 \\
betw\_5000 & 625 & 160638 & 147249 & 69251 & 226304 & 157053 & 0.00e+00 & 825275 \\
betw\_10000 & 625 & 1020000 & 807722 & 431800 & 1398694 & 966894 & 0.00e+00 & 7732552 \\
betw\_wt\_500 & 625 & 11 & 6.850 & 3.121 & 16 & 13 & 0.00e+00 & 74 \\
betw\_wt\_1000 & 625 & 95 & 65 & 30 & 131 & 101 & 0.00e+00 & 595 \\
betw\_wt\_2000 & 625 & 753 & 624 & 303 & 1071 & 768 & 0.00e+00 & 3666 \\
betw\_wt\_5000 & 625 & 11554 & 10881 & 5387 & 16402 & 11016 & 0.00e+00 & 43901 \\
betw\_wt\_10000 & 625 & 82753 & 71446 & 35153 & 119982 & 84829 & 0.00e+00 & 518176 \\
NACH\_500 & 625 & 0.364 & 0.397 & 0.306 & 0.457 & 0.150 & 0.00e+00 & 0.585 \\
NACH\_1000 & 625 & 0.462 & 0.495 & 0.421 & 0.545 & 0.124 & 0.00e+00 & 0.644 \\
NACH\_2000 & 625 & 0.529 & 0.551 & 0.497 & 0.599 & 0.102 & 0.00e+00 & 0.821 \\
NACH\_5000 & 625 & 0.585 & 0.594 & 0.553 & 0.636 & 8.31e-02 & 0.00e+00 & 1.182 \\
NACH\_10000 & 625 & 0.605 & 0.608 & 0.568 & 0.648 & 7.91e-02 & 0.00e+00 & 2.231 \\
\bottomrule
\end{tabular}

\end{table}

\begin{table}[htb]
  \centering
  \small
  \caption{Descriptive statistics for zone-averaged angular centrality measures.}\label{tab:desc_zone_centrality_ang}
  \begin{tabular}{lrlllllll}
\toprule
 & N & Mean & Median & Q1 & Q3 & IQR & Min & Max \\
\midrule
far\_500\_ang & 625 & 209 & 169 & 100 & 293 & 194 & 0.173 & 843 \\
far\_1000\_ang & 625 & 1129 & 1004 & 582 & 1513 & 931 & 5.767 & 3778 \\
far\_2000\_ang & 625 & 6186 & 6027 & 3766 & 8519 & 4753 & 26 & 15161 \\
far\_5000\_ang & 625 & 53368 & 55935 & 32839 & 71294 & 38455 & 528 & 105116 \\
far\_10000\_ang & 625 & 359098 & 376124 & 300312 & 423326 & 123014 & 21766 & 563114 \\
far\_norm\_500\_ang & 625 & 2.909 & 2.898 & 2.546 & 3.255 & 0.709 & 0.410 & 4.621 \\
far\_norm\_1000\_ang & 625 & 4.514 & 4.474 & 3.965 & 5.019 & 1.054 & 1.295 & 7.673 \\
far\_norm\_2000\_ang & 625 & 7.089 & 6.992 & 6.265 & 7.696 & 1.432 & 3.550 & 13 \\
far\_norm\_5000\_ang & 625 & 12 & 12 & 11 & 13 & 1.931 & 8.553 & 20 \\
far\_norm\_10000\_ang & 625 & 19 & 18 & 17 & 20 & 2.986 & 13 & 34 \\
closeness\_500\_ang & 625 & 9.77e-02 & 8.46e-03 & 4.41e-03 & 2.22e-02 & 1.78e-02 & 1.25e-03 & 23 \\
closeness\_1000\_ang & 625 & 7.59e-03 & 1.22e-03 & 7.39e-04 & 3.28e-03 & 2.54e-03 & 2.74e-04 & 0.437 \\
closeness\_2000\_ang & 625 & 7.36e-04 & 1.85e-04 & 1.25e-04 & 3.83e-04 & 2.58e-04 & 6.81e-05 & 2.65e-02 \\
closeness\_5000\_ang & 625 & 3.55e-05 & 1.85e-05 & 1.43e-05 & 3.23e-05 & 1.80e-05 & 9.56e-06 & 1.59e-03 \\
closeness\_10000\_ang & 625 & 3.40e-06 & 2.67e-06 & 2.37e-06 & 3.38e-06 & 1.01e-06 & 1.78e-06 & 9.74e-05 \\
close\_N1\_500\_ang & 625 & 0.470 & 0.374 & 0.329 & 0.448 & 0.119 & 0.226 & 23 \\
close\_N1\_1000\_ang & 625 & 0.249 & 0.237 & 0.209 & 0.268 & 5.90e-02 & 0.130 & 1.431 \\
close\_N1\_2000\_ang & 625 & 0.150 & 0.148 & 0.134 & 0.164 & 2.99e-02 & 8.24e-02 & 0.290 \\
close\_N1\_5000\_ang & 625 & 8.55e-02 & 8.55e-02 & 7.89e-02 & 9.27e-02 & 1.38e-02 & 5.33e-02 & 0.126 \\
close\_N1\_10000\_ang & 625 & 5.52e-02 & 5.48e-02 & 5.08e-02 & 6.00e-02 & 9.23e-03 & 3.09e-02 & 7.70e-02 \\
close\_N1.2\_500\_ang & 625 & 0.918 & 0.855 & 0.748 & 0.956 & 0.207 & 0.365 & 23 \\
close\_N1.2\_1000\_ang & 625 & 0.687 & 0.687 & 0.593 & 0.767 & 0.174 & 0.287 & 1.896 \\
close\_N1.2\_2000\_ang & 625 & 0.555 & 0.564 & 0.489 & 0.623 & 0.134 & 0.246 & 0.852 \\
close\_N1.2\_5000\_ang & 625 & 0.450 & 0.461 & 0.396 & 0.511 & 0.114 & 0.143 & 0.616 \\
close\_N1.2\_10000\_ang & 625 & 0.396 & 0.398 & 0.358 & 0.440 & 8.16e-02 & 0.106 & 0.565 \\
close\_N2\_500\_ang & 625 & 24 & 22 & 14 & 33 & 19 & 0.400 & 74 \\
close\_N2\_1000\_ang & 625 & 55 & 52 & 31 & 76 & 45 & 0.652 & 153 \\
close\_N2\_2000\_ang & 625 & 127 & 121 & 74 & 178 & 104 & 1.090 & 298 \\
close\_N2\_5000\_ang & 625 & 389 & 383 & 229 & 544 & 315 & 3.348 & 797 \\
close\_N2\_10000\_ang & 625 & 1099 & 1104 & 887 & 1388 & 501 & 21 & 1928 \\
harmonic\_500\_ang & 625 & 20 & 19 & 11 & 27 & 16 & 9.53e-02 & 60 \\
harmonic\_1000\_ang & 625 & 51 & 48 & 30 & 70 & 40 & 0.383 & 141 \\
harmonic\_2000\_ang & 625 & 128 & 121 & 76 & 176 & 100 & 0.718 & 291 \\
harmonic\_5000\_ang & 625 & 410 & 402 & 250 & 566 & 316 & 2.496 & 820 \\
harmonic\_10000\_ang & 625 & 1204 & 1222 & 969 & 1517 & 548 & 15 & 2067 \\
betw\_500\_ang & 625 & 134 & 88 & 39 & 187 & 148 & 0.00e+00 & 741 \\
betw\_1000\_ang & 625 & 990 & 736 & 319 & 1338 & 1018 & 0.00e+00 & 5380 \\
betw\_2000\_ang & 625 & 6974 & 5620 & 2608 & 9703 & 7095 & 0.00e+00 & 31342 \\
betw\_5000\_ang & 625 & 85320 & 57811 & 26618 & 124681 & 98063 & 0.00e+00 & 517862 \\
betw\_10000\_ang & 625 & 653772 & 413533 & 190716 & 823699 & 632983 & 0.00e+00 & 6203538 \\
NACH\_500\_ang & 625 & 0.656 & 0.726 & 0.589 & 0.812 & 0.223 & 0.00e+00 & 0.933 \\
NACH\_1000\_ang & 625 & 0.749 & 0.797 & 0.707 & 0.864 & 0.157 & 0.00e+00 & 1.059 \\
NACH\_2000\_ang & 625 & 0.789 & 0.817 & 0.749 & 0.876 & 0.127 & 0.00e+00 & 1.123 \\
NACH\_5000\_ang & 625 & 0.810 & 0.818 & 0.761 & 0.870 & 0.110 & 0.00e+00 & 1.381 \\
NACH\_10000\_ang & 625 & 0.826 & 0.824 & 0.774 & 0.877 & 0.104 & 0.00e+00 & 2.341 \\
\bottomrule
\end{tabular}

\end{table}

\subsubsection{Spatial Autocorrelation and Bootstrap Results}

Tables~\ref{tab:morans_i}--\ref{tab:bootstrap_ci} present spatial autocorrelation results. Moran's $I$ uses $k$-nearest neighbour weights, with $k$ derived from median network density at each distance threshold. Effective sample size ($N_{\text{eff}}$) accounts for spatial dependence via $N_{\text{eff}} \approx N(1-I)/(1+I)$. Block bootstrap confidence intervals (95\%) resample 100 spatial clusters with replacement (1,000 iterations) to preserve local spatial structure.

\begin{table}[htb]
  \centering
  \small
  \caption{Moran's $I$ spatial autocorrelation for centrality measures. Higher values indicate stronger positive spatial clustering; $k$ is the number of nearest neighbours used in the weights matrix.}\label{tab:morans_i}
  \begin{tabular}{lrrllll}
\toprule
Variable & Distance (m) & k (median density) & Moran's I & z-score & p (analytic) & p (permutation) \\
\midrule
closeness\_500 & 500 & 86 & 0.2095 & 294.77 & 0 & 0.001 \\
close\_N1\_500 & 500 & 86 & 0.1800 & 253.26 & 0 & 0.001 \\
close\_N1.2\_500 & 500 & 86 & 0.5277 & 742.31 & 0 & 0.001 \\
close\_N2\_500 & 500 & 86 & 0.8398 & 1181.44 & 0 & 0.001 \\
harmonic\_500 & 500 & 86 & 0.7971 & 1121.34 & 0 & 0.001 \\
gravity\_500 & 500 & 86 & 0.7855 & 1105.01 & 0 & 0.001 \\
betw\_500 & 500 & 86 & 0.4641 & 652.90 & 0 & 0.001 \\
betw\_wt\_500 & 500 & 86 & 0.4695 & 660.57 & 0 & 0.001 \\
NACH\_500 & 500 & 86 & 0.2848 & 400.63 & 0 & 0.001 \\
closeness\_1000 & 1000 & 330 & 0.1281 & 354.90 & 0 & 0.001 \\
close\_N1\_1000 & 1000 & 330 & 0.1877 & 519.97 & 0 & 0.001 \\
close\_N1.2\_1000 & 1000 & 330 & 0.4227 & 1170.93 & 0 & 0.001 \\
close\_N2\_1000 & 1000 & 330 & 0.7933 & 2197.63 & 0 & 0.001 \\
harmonic\_1000 & 1000 & 330 & 0.7444 & 2062.24 & 0 & 0.001 \\
gravity\_1000 & 1000 & 330 & 0.7216 & 1998.93 & 0 & 0.001 \\
betw\_1000 & 1000 & 330 & 0.3373 & 934.31 & 0 & 0.001 \\
betw\_wt\_1000 & 1000 & 330 & 0.3583 & 992.73 & 0 & 0.001 \\
NACH\_1000 & 1000 & 330 & 0.1367 & 378.82 & 0 & 0.001 \\
closeness\_2000 & 2000 & 1199 & 0.0641 & 342.97 & 0 & 0.001 \\
close\_N1\_2000 & 2000 & 1199 & 0.1474 & 788.16 & 0 & 0.001 \\
close\_N1.2\_2000 & 2000 & 1199 & 0.3487 & 1863.65 & 0 & 0.001 \\
close\_N2\_2000 & 2000 & 1199 & 0.7303 & 3903.16 & 0 & 0.001 \\
harmonic\_2000 & 2000 & 1199 & 0.6657 & 3558.00 & 0 & 0.001 \\
gravity\_2000 & 2000 & 1199 & 0.6271 & 3351.90 & 0 & 0.001 \\
betw\_2000 & 2000 & 1199 & 0.1814 & 969.47 & 0 & 0.001 \\
betw\_wt\_2000 & 2000 & 1199 & 0.2107 & 1126.41 & 0 & 0.001 \\
NACH\_2000 & 2000 & 1199 & 0.0610 & 325.95 & 0 & 0.001 \\
\bottomrule
\end{tabular}

\end{table}

\begin{table}[htb]
  \centering
  \small
  \caption{Effective sample size ($N_{\text{eff}}$) by centrality measure and distance threshold. Well-normalised measures with high spatial coherence have lower $N_{\text{eff}}$ due to positive autocorrelation.}\label{tab:neff}
  \begin{tabular}{lrlll}
\toprule
Variable & Distance (m) & Moran's I & Neff & Neff/N \\
\midrule
closeness\_500 & 500 & 0.2095 & 27,558 & 65.4\% \\
close\_N1\_500 & 500 & 0.1800 & 29,301 & 69.5\% \\
close\_N1.2\_500 & 500 & 0.5277 & 13,037 & 30.9\% \\
close\_N2\_500 & 500 & 0.8398 & 3,671 & 8.7\% \\
harmonic\_500 & 500 & 0.7971 & 4,761 & 11.3\% \\
gravity\_500 & 500 & 0.7855 & 5,066 & 12.0\% \\
betw\_500 & 500 & 0.4641 & 15,434 & 36.6\% \\
betw\_wt\_500 & 500 & 0.4695 & 15,220 & 36.1\% \\
NACH\_500 & 500 & 0.2848 & 23,474 & 55.7\% \\
closeness\_1000 & 1000 & 0.1281 & 32,591 & 77.3\% \\
close\_N1\_1000 & 1000 & 0.1877 & 28,840 & 68.4\% \\
close\_N1.2\_1000 & 1000 & 0.4227 & 17,111 & 40.6\% \\
close\_N2\_1000 & 1000 & 0.7933 & 4,860 & 11.5\% \\
harmonic\_1000 & 1000 & 0.7444 & 6,178 & 14.7\% \\
gravity\_1000 & 1000 & 0.7216 & 6,819 & 16.2\% \\
betw\_1000 & 1000 & 0.3373 & 20,898 & 49.6\% \\
betw\_wt\_1000 & 1000 & 0.3583 & 19,919 & 47.2\% \\
NACH\_1000 & 1000 & 0.1367 & 32,023 & 75.9\% \\
closeness\_2000 & 2000 & 0.0641 & 37,083 & 87.9\% \\
close\_N1\_2000 & 2000 & 0.1474 & 31,330 & 74.3\% \\
close\_N1.2\_2000 & 2000 & 0.3487 & 20,364 & 48.3\% \\
close\_N2\_2000 & 2000 & 0.7303 & 6,573 & 15.6\% \\
harmonic\_2000 & 2000 & 0.6657 & 8,463 & 20.1\% \\
gravity\_2000 & 2000 & 0.6271 & 9,662 & 22.9\% \\
betw\_2000 & 2000 & 0.1814 & 29,219 & 69.3\% \\
betw\_wt\_2000 & 2000 & 0.2107 & 27,488 & 65.2\% \\
NACH\_2000 & 2000 & 0.0610 & 37,321 & 88.5\% \\
\bottomrule
\end{tabular}

\end{table}

\begin{table}[htb]
  \centering
  \small
  \caption{Block bootstrap Spearman correlations ($\rho$) between centrality measures and PCA~1 (landuse intensity), with 95\% confidence intervals. Negative correlations for unnormalized \emph{Closeness} indicate pathological behaviour; near-zero correlations for \emph{Normalised Closeness} indicate ineffective normalisation.}\label{tab:bootstrap_ci}
  \begin{tabular}{lrlllll}
\toprule
Variable & Distance (m) & $\rho$ & CI Low & CI High & Moran's I & Neff \\
\midrule
closeness\_500 & 500 & -0.704 & -0.756 & -0.641 & 0.2095 & 27,558 \\
close\_N1\_500 & 500 & -0.114 & -0.158 & -0.066 & 0.1800 & 29,301 \\
close\_N1.2\_500 & 500 & 0.503 & 0.425 & 0.565 & 0.5277 & 13,037 \\
close\_N2\_500 & 500 & 0.667 & 0.597 & 0.724 & 0.8398 & 3,671 \\
harmonic\_500 & 500 & 0.646 & 0.574 & 0.704 & 0.7971 & 4,761 \\
gravity\_500 & 500 & 0.625 & 0.553 & 0.685 & 0.7855 & 5,066 \\
betw\_500 & 500 & 0.576 & 0.524 & 0.621 & 0.4641 & 15,434 \\
betw\_wt\_500 & 500 & 0.547 & 0.490 & 0.594 & 0.4695 & 15,220 \\
NACH\_500 & 500 & 0.516 & 0.467 & 0.560 & 0.2848 & 23,474 \\
closeness\_1000 & 1000 & -0.759 & -0.804 & -0.704 & 0.1281 & 32,591 \\
close\_N1\_1000 & 1000 & -0.002 & -0.072 & 0.085 & 0.1877 & 28,840 \\
close\_N1.2\_1000 & 1000 & 0.613 & 0.537 & 0.675 & 0.4227 & 17,111 \\
close\_N2\_1000 & 1000 & 0.771 & 0.712 & 0.821 & 0.7933 & 4,860 \\
harmonic\_1000 & 1000 & 0.744 & 0.682 & 0.797 & 0.7444 & 6,178 \\
gravity\_1000 & 1000 & 0.729 & 0.666 & 0.781 & 0.7216 & 6,819 \\
betw\_1000 & 1000 & 0.591 & 0.548 & 0.631 & 0.3373 & 20,898 \\
betw\_wt\_1000 & 1000 & 0.590 & 0.543 & 0.633 & 0.3583 & 19,919 \\
NACH\_1000 & 1000 & 0.502 & 0.462 & 0.538 & 0.1367 & 32,023 \\
closeness\_2000 & 2000 & -0.770 & -0.820 & -0.705 & 0.0641 & 37,083 \\
close\_N1\_2000 & 2000 & 0.148 & 0.054 & 0.242 & 0.1474 & 31,330 \\
close\_N1.2\_2000 & 2000 & 0.694 & 0.618 & 0.752 & 0.3487 & 20,364 \\
close\_N2\_2000 & 2000 & 0.825 & 0.777 & 0.864 & 0.7303 & 6,573 \\
harmonic\_2000 & 2000 & 0.825 & 0.776 & 0.866 & 0.6657 & 8,463 \\
gravity\_2000 & 2000 & 0.816 & 0.764 & 0.858 & 0.6271 & 9,662 \\
betw\_2000 & 2000 & 0.536 & 0.492 & 0.578 & 0.1814 & 29,219 \\
betw\_wt\_2000 & 2000 & 0.576 & 0.532 & 0.617 & 0.2107 & 27,488 \\
NACH\_2000 & 2000 & 0.438 & 0.402 & 0.472 & 0.0610 & 37,321 \\
\bottomrule
\end{tabular}

\end{table}
